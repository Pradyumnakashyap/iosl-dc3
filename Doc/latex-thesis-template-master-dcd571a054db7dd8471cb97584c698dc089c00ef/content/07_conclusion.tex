\chapter{Conclusion}
\label{cha:conclusion}

Primarily a brief explanation stating the requirement was provided to the team which basically was to create a dashboard for monitoring logistics data.
The purpose of DC3 as a web application is to bring together three different user groups so that they could use a single application to monitor logistics. As per the requirements it was clear to is that the web application is supposedly used by three major user groups and hence we decided to go on and create three personas for our project.The primary focus was to create a web based application with a look and feel that matches that of the template provided to us and make sure all the functionalities provided were to be included in the application. Post understanding the basic frameworks neccesary to develop the application, we dived into the development process.The involvement of a Back-end in our case is to give completeness to the website by providing a Database for saving the user information, history of registered/delivered packages and saving incidents that occur in the middle of transit.Then came the development phase where we  actually had plenty of hit and trials; adding and removal of components; re-designing old components; adjusting the size or placement of these components on the dashboard. Also the divison of work based on personas also made sure that each user group has a different view of the dashboard whist keeping a consistent outlook with the SPA approach. Further the React JavaScript being popular for its component based development actually made things more convenient to embrace the hit and trial approach of work because it made us obligated to stick on to the component based development which made the code more re-usable and also allowed us to immediately view our changes in the running local servers. In the DC3 application the presence of a NavBar component provides a rigid classification of different components to our application; for instance "Register.js" component is kept under the Package Management category and "Incident.js" is kept under Incident Management category which ensures that not all components are refreshed or in this case re-rendered while navigating from one functional component to another. When we consider the blockchain setup, ideally most of the data is to be derived by the DC3 application comes from a dedicated blockchain that constantly provides the data related to the current condition of a package and part of this data are sent by the temperature and pressure sensors allocated to a package as per customer request. This step is to be taken care of as part of the future work as the blockchain set up is yet to be completed. With inclusion of the same, the DC3 when compared to the traditional tracking services provided by the Logistic companies, gives a much more broader view of tracking and monitoring with advanced functionalities that can also serve multiple logistic companies simultaneously. Considering a bigger picture, DC3 outsmarts the classical approach of logistics monitoring by providing to the users as well as the companies at the same time; It is a single platform that is capable of displaying information of registered packages that are handled by more than one company during its transit across the continent.
Finally with the diversity among our team members coming from different areas of expertise, it was not an easy task to immediately start off with the project development. We needed a few weeks to introduce ourselves to React JS platform and once familiar started developing. we were very glad as this turned out to be a huge learning curve for all the team members in terms of front end development and project management skills.We also learned as a team the importance of teamwork, regular brainstorming, collective research sessions in developing an application on the whole.Also regular meetings with our supervisor did help us a great deal to set weekly targets to be achieved in terms of the project completion. Having these smaller targets set on weekly basis kept us motivated and helped complete the project in time.Overall the knowledge and experience gained from this project can indeed be very helpful to all of us in pursuing a career in the Information Technology sector. 
