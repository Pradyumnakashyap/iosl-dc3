\chapter{Conclusion}
\label{cha:conclusion}

In the beginning we were given a brief idea about the project explaining that it is a dashboard for monitoring logistics data. As per the requirements it was clear that the web application is supposed to be used by three different user groups. Hence the primary focus was on the look and feel of the landing screen, navigation pane and dashboard components. Development in this part involved plenty of hit and trials; adding and removing components; re-designing old components, changing and adjusting size and placement of components on the dashboard therefore making sure that each user group has a different view of the dashboard. Using reach JavaScript makes it very convenient to embrace the hit and trial approach of work because we are able to view the changes immediately into the running local server, once the code changes are saved in VS Code. The presence of NavBar component provides a rigid classification of different components; for example Register.js is kept under Package Management category and Incident.js is kept under Incident Management category. This ensures the whole webpage does not get refreshed while navigating from one functional component to another. The purpose of DC3 as a web application is to bring together three different user groups so that they could use a single application to monitor logistics. The involvement of a Back-end is to give completeness to the website by providing a Database for saving the user information, history of registered/delivered packages and saving incidents that occur in the middle of transit. Otherwise the min data derived by DC3 is from a dedicated Blockchain that constantly provides the data related to the current condition of a package and part of these data are sent by the temperature and pressure sensors allocated to a package as per Customer request. As compared to the traditional tracking services provided by Logistic companies, DC3 gives a much more broader view of tracking and monitoring with advanced functionalities that can also serve multiple logistic companies simultaneously. In the big picture, DC3 outsmarts the classical approach of logistics monitoring by providing the users; one single platform that is capable of displaying information of registered packages that are handled by more than one company during its long transit across the continent. Considering the diversity among our team members in knowledge background and areas of expertise, it was not an easy job to immediately start working on the project development. We needed few weeks to introduce ourselves to React JS and once we started developing the web pages, we learned a lot about this technology while fiddling with the individual functionalities of every component. We have also come across several challenges during the development phase that we have overcome with teamwork, brainstorming, collective research and individual hard work.This knowledge and experience we have gained while developing DC3 dashboard is very priceless and some of us even look forward to pursue this track of full stack development with React/Node JavaScript and make it an integral part of our careers in Information Technology sector. 
